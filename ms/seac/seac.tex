% interactcadsample.tex
% v1.03 - April 2017

\documentclass[]{interact}

\usepackage{epstopdf}% To incorporate .eps illustrations using PDFLaTeX, etc.
\usepackage{subfigure}% Support for small, `sub' figures and tables
%\usepackage[nolists,tablesfirst]{endfloat}% To `separate' figures and tables from text if required

\usepackage{natbib}% Citation support using natbib.sty
\bibpunct[, ]{(}{)}{;}{a}{}{,}% Citation support using natbib.sty
\renewcommand\bibfont{\fontsize{10}{12}\selectfont}% Bibliography support using natbib.sty

\theoremstyle{plain}% Theorem-like structures provided by amsthm.sty
\newtheorem{theorem}{Theorem}[section]
\newtheorem{lemma}[theorem]{Lemma}
\newtheorem{corollary}[theorem]{Corollary}
\newtheorem{proposition}[theorem]{Proposition}

\theoremstyle{definition}
\newtheorem{definition}[theorem]{Definition}
\newtheorem{example}[theorem]{Example}

\theoremstyle{remark}
\newtheorem{remark}{Remark}
\newtheorem{notation}{Notation}

% see https://stackoverflow.com/a/47122900

% Pandoc citation processing

\usepackage{hyperref}
\usepackage[utf8]{inputenc}
\def\tightlist{}


\begin{document}

\articletype{ORIGINAL RESEARCH ARTICLE}

\title{Morphological trajectories suggest significant changes in
preference and design intent associated with Gahagan bifaces from Caddo
burials}


\author{\name{Robert Z. Selden, Jr.$^{a}$, John E. Dockall$^{b}$}
\affil{$^{a}$Heritage Research Center, Stephen F. Austin State
University; Department of Biology, Stephen F. Austin State University;
Cultural Heritage Department, Jean Monnet
University; $^{b}$Cox\textbar McClain Environmental Consultants, Inc.}
}

\thanks{CONTACT Robert Z. Selden,
Jr.. Email: \href{mailto:zselden@sfasu.edu}{\nolinkurl{zselden@sfasu.edu}}, John
E.
Dockall. Email: \href{mailto:johnd@coxmcclain.com}{\nolinkurl{johnd@coxmcclain.com}}}

\maketitle

\begin{abstract}
Gahagan bifaces represent one of three categories of Caddo material
culture that express significant differences in morphology across the
same geography as Caddo bottles and Perdiz arrow points, and has also
been found to differ significantly between the ancestral Caddo area and
central Texas. This study asks whether Gahagan biface morphology differs
between stratigraphically-defined chronological contexts at the Mounds
Plantation and George C. Davis sites, and whether Gahagan biface
morphology might differ in morphology based on differences in Caddo
burial practices. Results indicate a significant--and
inverse--difference in size between burial contexts at both Mounds
Plantation and George C. Davis. At both sites, shape remains consistent
and does not differ among contexts, indicating an established
\texttt{shape\ preference} that may have shifted in size due to cyclical
differences in the variable social mechanisms associated with raw
material procurement. Gahagan bifaces also differ in shape between
burial contexts where a biface was placed \emph{alongside an individual}
and those found as part of a cache \emph{alongside the northern wall of
the burial feature}. Each burial tradition articulates with a distinct
\texttt{community\ of\ practice} relating to Gahagan biface
\texttt{placement} and \texttt{design\ intent}.
\end{abstract}

\begin{keywords}
American Southeast; Caddo; NAGPRA; 3D geometric morphometrics; museum
studies; digital humanities
\end{keywords}

\hypertarget{introduction}{%
\section{Introduction}\label{introduction}}

Once upon a time\ldots{}

\hypertarget{acknowledgments}{%
\section*{Acknowledgments}\label{acknowledgments}}
\addcontentsline{toc}{section}{Acknowledgments}

We extend our gratitude to the Caddo Nation of Oklahoma, the Williamson
Museum at Northwestern State University, the Louisiana State Exhibit
Museum, the Texas Archeological Research Laboratory at The University of
Texas at Austin, the Brazos Valley Museum of Natural History, the Texas
Parks and Wildlife Department, and the Sam Noble Oklahoma Museum of
Natural Science for the requisite permissions and access needed to
generate 3D scans of the Gahagan bifaces. Thanks to Harry J. Shafer,
Hiram F. (Pete) Gregory, Christian S. Hoggard, and David K. Thulman for
their comments on the analyses of Gahagan biface shape.

RZS extends his gratitude to Christian S. Hoggard and David K. Thulman
for their thoughtful comments and constructive criticisms of the
landmarking protocol used in this study
(\href{https://github.com/aksel-blaise/gahaganmorph2/blob/master/analysis/landmarking-protocol.md}{\texttt{LM3d1}}),
as well as the landmarking protocol for Gahagan bifaces that will be
used in the next iteration of these analytical efforts
(\href{https://seldenlab.github.io/gahaganmorph.3/landmarking-protocol-3d2.html}{\texttt{LM3d2}});
to Martin Hinz for fielding questions related to the \texttt{oxcAAR}
package, to Derek Hamilton for his guidance with the chronological
models; and to Dean C. Adams, Michael L. Collyer, Emma Sherratt, Lauren
Butaric, and Kersten Bergstrom for their constructive criticisms,
general comments, and suggestions throughout the development of this
research program.

\hypertarget{funding}{%
\section*{Funding}\label{funding}}
\addcontentsline{toc}{section}{Funding}

Components of this analytical work flow were developed and funded by a
Preservation Technology and Training grant (P14AP00138) to RZS from the
National Center for Preservation Technology and Training (NCPTT), and
additional grants to RZS from the Caddo Tribe of Oklahoma, National
Forests and Grasslands in Texas (15-PA-11081300-033) and the United
States Forest Service (20-PA-11081300-074). Funding to scan the Gahagan
bifaces at the Williamson Museum at Northwestern State University,
Louisiana State Exhibit Museum, Texas Archeological Research Laboratory
at The University of Texas at Austin, and Sam Noble Oklahoma Museum of
Natural Science was provided to the RZS by the Heritage Research Center
at Stephen F. Austin State University.

\hypertarget{data-management}{%
\section*{Data management}\label{data-management}}
\addcontentsline{toc}{section}{Data management}

The analysis code associated with this project can be accessed through
this document or the
\href{https://github.com/seldenlab/gahaganmorph.3}{GitHub} repository,
which is digitally curated on the Open Science Framework
\href{https://osf.io/y7b39/}{DOI: 10.17605/OSF.IO/Y7B39}. The
reproducible nature of this undertaking provides a means for others to
critically assess and evaluate the various analytical components
\citep{RN20915, RN20916, RN20917}, which is a necessary requirement for
the production of reliable knowledge.

Reproducibility projects in \href{https://osf.io/ezcuj/}{psychology} and
\href{https://www.cos.io/rpcb}{cancer biology} are impacting current
research practices across all domains. Examples of reproducible research
are becoming more abundant in archaeology
\citep{RN20804, RN21009, RN11783, RN21001, RN9364}, and the next
generation of archaeologists are learning those tools and methods needed
to reproduce and/or replicate research results \citep{RN21007}.
Reproducible and replicable research work flows are often employed at
the highest levels of humanities-based inquiries to mitigate concern or
doubt regarding proper execution, and is of particular import should the
results have---explicitly or implicitly---a major impact on scientific
progress \citep{RN21008}.

\bibliographystyle{tfcad}
\bibliography{interactcadsample.bib}


\input{"appendix.tex"}


\end{document}
