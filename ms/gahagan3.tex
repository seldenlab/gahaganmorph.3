% interactcadsample.tex
% v1.03 - April 2017

\documentclass[]{interact}

\usepackage{epstopdf}% To incorporate .eps illustrations using PDFLaTeX, etc.
\usepackage{subfigure}% Support for small, `sub' figures and tables
%\usepackage[nolists,tablesfirst]{endfloat}% To `separate' figures and tables from text if required

\usepackage{natbib}% Citation support using natbib.sty
\bibpunct[, ]{(}{)}{;}{a}{}{,}% Citation support using natbib.sty
\renewcommand\bibfont{\fontsize{10}{12}\selectfont}% Bibliography support using natbib.sty

\theoremstyle{plain}% Theorem-like structures provided by amsthm.sty
\newtheorem{theorem}{Theorem}[section]
\newtheorem{lemma}[theorem]{Lemma}
\newtheorem{corollary}[theorem]{Corollary}
\newtheorem{proposition}[theorem]{Proposition}

\theoremstyle{definition}
\newtheorem{definition}[theorem]{Definition}
\newtheorem{example}[theorem]{Example}

\theoremstyle{remark}
\newtheorem{remark}{Remark}
\newtheorem{notation}{Notation}

% see https://stackoverflow.com/a/47122900

% Pandoc citation processing

\usepackage{hyperref}
\usepackage[utf8]{inputenc}
\def\tightlist{}


\begin{document}

\articletype{CHAPTER X}

\title{Morphological trajectories suggest differential preference and
design intent associated with Gahagan bifaces from Caddo burials in the
American Southeast}


\author{\name{Robert Z. Selden, Jr.$^{a}$, John E. Dockall$^{b}$}
\affil{$^{a}$Heritage Research Center, Stephen F. Austin State
University; Department of Biology, Stephen F. Austin State University;
Cultural Heritage Department, Jean Monnet
University; $^{b}$Cox\textbar McClain Environmental Consultants, Inc.}
}

\thanks{CONTACT Robert Z. Selden,
Jr.. Email: \href{mailto:zselden@sfasu.edu}{\nolinkurl{zselden@sfasu.edu}}, John
E.
Dockall. Email: \href{mailto:johnd@coxmcclain.com}{\nolinkurl{johnd@coxmcclain.com}}}

\maketitle

\begin{abstract}
Gahagan bifaces differ in morphology across the same geography as Caddo
bottles and Perdiz arrow points, and between the Caddo and central Texas
regions. This study asks whether Gahagan biface morphology differs
between stratigraphically-defined---and chronologically
discrete---burial contexts at the Mounds Plantation and George C. Davis
sites, and whether Gahagan biface morphology may differ based on
qualitative differences in Caddo burial practices. Results indicate a
significant difference in Gahagan biface size between stratigraphic
contexts at the Mounds Plantation site. At both Mounds Plantation and
George C. Davis, biface shape remains consistent and does not differ
among contexts, indicating an established \texttt{shape\ preference} in
the northern and southern behavioral regions that may have shifted
significantly in size at Mounds Plantation due to cyclical differences
in the variable social mechanisms associated with raw material
procurement. Gahagan bifaces also differ in shape between Caddo burial
contexts where a biface was placed \emph{atop or alongside an
individual} and those found as part of a cache placed \emph{alongside
the northern wall of the burial feature}. Each burial tradition
articulates with a distinct \texttt{community\ of\ practice} relating to
Gahgan biface placement and design intent.
\end{abstract}

\begin{keywords}
Caddo; NAGPRA; archaeoinformatics; 3D geometric morphometrics; museum
studies; digital humanities; socio-ecological systems
\end{keywords}

\begin{quote}
Studying the history of paper exposes a number of historical
misconceptions, the most important of which is this technological
fallacy: the idea that technology changes society. It is exactly the
reverse. Society develops technology to address the chages that are
taking place within it \citep[xiv]{RN10878}.
\end{quote}

\hypertarget{introduction}{%
\section{Introduction}\label{introduction}}

Found to occur along the western margin of the American Southeast,
Gahagan bifaces are among the most recognizable and well known
components of Caddo material culture, and were regularly included with
burial offerings during the Formative (CE 800-1000) and Early (CE
1000-1250) Caddo periods
\citep{RN7115,RN8189,RN5746,RN8186,RN8174,RN8176}. Recent efforts to
characterize general trends associated with Gahagan biface morphology
yielded support for a \texttt{spatial\ boundary} that divides the
northern and southern Caddo behavioral regions based upon morphological
differences that occur in Caddo bottles, Gahagan bifaces, and Perdiz
arrow points (\textbf{\texttt{Chapter\ X,\ Figure\ 1c}})
\citep{RN7925,RN8071,RN8361,RN8967,RN11064,RN8154}. A second
\texttt{shape\ boundary} has also been posited based upon differences in
Gahagan biface morphology between the Caddo and central Texas regions
\citep{RN8318}.

\begin{figure}

{\centering \includegraphics[width=0.8\linewidth]{img/fig02} 

}

\caption{Gahagan bifaces from the northern and southern Caddo behavioral regions. Bifaces recovered atop or alongside an individual denoted by black dot. For reference to scale, the Gahagan biface at bottom left (m3) measures 48cm in length. Additional information for each biface, including the option to download full-resolution 2D images of individual bifaces, can be found at https://scholarworks.sfasu.edu/ita-gahaganbiface/.}\label{fig:gahagan bifaces 2D}
\end{figure}

\hypertarget{hypotheses}{%
\subsection{Hypotheses}\label{hypotheses}}

\hypertarget{hypothesis-1}{%
\subsubsection{Hypothesis 1}\label{hypothesis-1}}

\textbf{Gahagan bifaces selected for inclusion in Caddo burials differ
in morphology through time.}

Hypothesis 1 is tested using two samples of Gahagan bifaces from Caddo
burial contexts at the Mounds Plantation and George C. Davis sites where
stratigraphy dictates differing temporal positions. There is an
important cultural distinction between archaeological contexts at Mounds
Plantation and George C. Davis, in that all of the Gahagan bifaces from
Mounds Plantation were recovered in association with an individual
(Burials 1, 2, 5, and 8) \citep{RN8174}, and all but four of the Gahagan
bifaces from George C. Davis articulate with caches included along the
northern periphery of two group burials (Features 119 and 134)
\citep{RN5746,RN8186}. The information gathered from the analysis
provides valuable insights with regard to changes in aesthetic
preference, as well as design intent for Gahagan bifaces recovered from
discrete temporal intervals.

\hypertarget{hypothesis-2}{%
\subsubsection{Hypothesis 2}\label{hypothesis-2}}

\textbf{Gahagan bifaces included in Caddo burials as a cache differ in
morphology from those placed atop or alongside an individual.}

Hypothesis 2 is tested using Gahagan bifaces from two discrete Caddo
burial practices; one interred as part of a \emph{cache}, and the other
placed atop or alongside \emph{individuals}. Distinct Caddo burial
practices may have been constrained by local morphological requirements,
highlighting aspects of differential design intent.

\hypertarget{contexts-of-recovery}{%
\section{Contexts of recovery}\label{contexts-of-recovery}}

All Gahagan bifaces used in the analysis were recovered from Caddo
burial contexts excavated between 1912 and 1969. Stratigraphic contexts
were used to delimit temporal contexts---although the span of that
temporal difference remains unclear---and figures were redrafted from
the original works that illustrate clear stratigraphic differences
(Figure 3). Additional contextual information was used to identify
Gahagan bifaces placed atop or alongside an individual, or in caches
along the northern wall of the burial pit. Specific references are made
to published images and illustrations that depict these burial contexts
during excavation. While no images or illustrations of burial contexts
were included, direct references to the figures and illustrations are
included in the text and accompany the supplemental materials
\citep{RN11065}.

\begin{figure}

{\centering \includegraphics[width=0.8\linewidth]{img/fig03} 

}

\caption{Stratigraphic position of Burial Pits 1 and 2 at the Mounds Plantation site (a); adapted from Webb (1975:Figure 7), where Burial Pit 2 is the only burial found to be intrusive from the mound surface, and stratigraphic position of Features 119 and 134 at the George C. Davis site (b); adapted from Story (1997:Figure 13).}\label{fig:h1a}
\end{figure}

\hypertarget{mounds-plantation}{%
\subsection{Mounds Plantation}\label{mounds-plantation}}

In assessing a temporal change in preference between Gahagan bifaces
from Mound 5 at the Mounds Plantation site, those from burials included
during mound development and/or construction (Burial Pits 1, 5, and 8)
are contrasted with those from Burial Pit 2 which cuts into the corner
of Burial Pit 1, cutting downward from the mound's surface (see Figure
2:a\textsubscript{1}-l\textsubscript{1}) \citep{RN8174}. The
stratigraphic position of Burial Pit 2 indicates that this burial
occurred after those associated with Burial Pits 1, 5, and 8 (see Figure
3) \citep{RN8174}.

\hypertarget{gahagan-mound}{%
\subsection{Gahagan Mound}\label{gahagan-mound}}

Gahagan bifaces were identified in 1911 in Deposits 1, 2, and 3 of
Burial Pit 1 at the Gahagan Mound site in northwest Louisiana by
\citet[Figures 18-19, 21]{RN7115}. A subsequent excavation at the
Gahagan Mound site in 1936 identified two additional burial features
containing Gahagan bifaces (Burial Pits 2 and 3) (see Figure
2:a\textsubscript{2}-ee\textsubscript{2}) \citep[Plate 27]{RN8176}, and
the entirety of the Gahagan Mound site was eventually destroyed by the
meander of the Red River between four and five years later
\citep{RN10759}.

\hypertarget{george-c.-davis}{%
\subsection{George C. Davis}\label{george-c.-davis}}

To assess the temporal change in preference between caches of Gahagan
bifaces recovered from Mound C at the George C. Davis site, those from
Feature 134 are contrasted with those from Feature 119 (see Figure
2:a\textsubscript{3}-y\textsubscript{3}) \citep{RN5746, RN8186}. The
stratigraphic position of Feature 119 indicates that the burials in that
feature occurred subsequent to those associated with Feature 134 (see
Figure 3) \citep{RN5746, RN8186}.

\begin{figure}

{\centering \includegraphics[width=0.7\linewidth]{img/fig04} 

}

\caption{Network of associated diagnostic artifacts recovered with Gahagan bifaces. Nodes are sized by degree, meaning that the larger nodes articulate with a greater number of connections. Node color is defined by the northern (light blue) or southern (dark blue) behavioral regions. Edges are weighted by size and color, meaning that thicker and darker lines represent a greater number diagnostic artifacts that are co-present between contexts.}\label{fig:associated.net}
\end{figure}

\begin{table}
\tbl{Diagnostic artifact types from Caddo burials found in association with Gahagan bifaces.}
{\begin{tabular}{ll} \toprule
Contexts & Diagnostics \\
\midrule
16CD12-BP1 & AL, CA, CC, FR, HA, HE, HFE \\
16CD12-BP2 & AL, HA \\
16CD12-BP5 & AL, FR, HA, SC \\
16CD12-BP8 & CA, HE, HFE, SC \\
16RR1-BP2 & AL, C, HA, HFE, HR, KI, SC \\
16RR1-BP3 & AL, C, HFE, SC \\
41CE19-F119 & AL, C, HA, HFE \\
41CE19-F134 & AL, C, HA \\
\bottomrule
\end{tabular}}
\tabnote{Associated diagnostic artifacts from Caddo burial contexts at the Mounds Plantation (16CD12), Gahagan Mound (16RR1), and George C. Davis (41CE19) sites include Alba arrow points (AL), celts (C), Catahoula arrow points (CA), Coles Creek ceramics (CC), Friley arrow points (FR), Gahagan bifaces (GB), Hayes arrow points (HA), Hickory Engraved ceramics (HE), Holly Fine Engraved ceramics (HFE), Harrell arrow points (HR), Kiam Incised ceramics (KI), and Scallorn arrow points (SC).}
\label{sample-table}
\end{table}

\hypertarget{methods-and-results}{%
\section{Methods and results}\label{methods-and-results}}

Data collection procedures are outlined in \cite{RN8154} and
\cite{RN8318}. Characteristic points and tangents used in the
landmarking protocol were inspired by the work of \cite{RN5700}, and the
landmarking protocol is discussed in detail in the
\href{https://seldenlab.github.io/gahaganmorph.3/}{supplementary
materials}.

Landmarks were aligned to a global coordinate system
\citep{RN8102,RN8587,RN8384}, achieved through generalized Procrustes
superimposition \citep{RN8525}, performed in R 4.1.1 \citep{RN8584}
using the \texttt{geomorph} package v4.0.1 \citep{RN8565,RN9565}.
Procrustes superimposition translates, scales, and rotates coordinate
data allowing for comparisons among objects \citep{RN5698,RN8525}. The
\texttt{geomorph} package uses a partial Procrustes superimposition that
projects the aligned specimens into tangent space subsequent to
alignment in preparation for the use of multivariate methods that assume
linear space \citep{RN8511,RN8384}.

Principal components analysis \citep{RN8576,RN10875} was used to
visualize shape variation among the bifaces. The shape changes described
by each principal axis are commonly visualized using thin-plate spline
warping of the landmarks or the reference 3D mesh \citep{RN8555,RN8553}.
A residual randomization permutation procedure (RRPP; n = 10,000
permutations) was used for all Procrustes ANOVAs \citep{RN8579,RN8334},
which has higher statistical power and a greater ability to identify
patterns in the data should they be present \citep{RN6995}. To assess
whether shape differs by context, Procrustes ANOVAs \citep{RN7046} were
run that enlist effect-sizes (z-scores) computed as standard deviates of
the generated sampling distributions \citep{RN8477}.

A comparison of mean consensus configurations was used to characterize
shape variation in Gahagan bifaces recovered from contexts where they
were found atop or beside an individual or as part of a cache along the
northern wall of the burial pit.

\hypertarget{hypothesis-1a}{%
\subsection{Hypothesis 1a}\label{hypothesis-1a}}

In assessing a temporal change in preference between Gahagan bifaces
from Mound 5 at the Mounds Plantation site, those from burials included
during mound development and/or construction (Burial Pits 1, 5, and 8)
are contrasted with those from Burial Pit 2 which cuts into the corner
of Burial Pit 1, cutting downward from the mound's surface
\citep{RN8174}. The stratigraphic position of Burial Pit 2 indicates
that this burial occurred after those associated with Burial Pits 1, 5,
and 8 \citep{RN8174}.

A Procrustes ANOVA was used to test whether a significant difference
exists in Gahagan biface shape by context (RRPP = 10,000; Rsq = 0.1211;
Pr(\textgreater F) = 0.2224), followed by a second to assess differences
in Gahagan biface (centroid) size by context (RRPP = 10,000; Rsq =
0.49514; Pr(\textgreater F) = 0.0059).

\hypertarget{hypothesis-1b}{%
\subsection{Hypothesis 1b}\label{hypothesis-1b}}

To assess the temporal change in preference between caches of Gahagan
bifaces recovered from Mound C at the George C. Davis site, those from
Feature 134 are contrasted with those from Feature 119
\citep{RN5746,RN8186}. The stratigraphic position of Feature 119
indicates that the burials in that feature occurred subsequent to those
associated with Feature 134 \citep{RN5746,RN8186}.

A Procrustes ANOVA was used to test whether a significant difference
exists in Gahagan biface shape by context (RRPP = 10,000; Rsq = 0.04604;
Pr(\textgreater F) = 0.3061), followed by a second to assess differences
in Gahagan biface (centroid) size by context (RRPP = 10,000; Rsq =
0.14787; Pr(\textgreater F) = 0.0599).

\hypertarget{hypothesis-2-1}{%
\subsection{Hypothesis 2}\label{hypothesis-2-1}}

Hypothesis 2 is tested using two broad categories of Gahagan bifaces;
one interred as part of a \emph{cache}, and the other placed atop or
alongside \emph{individuals}. Distinct Caddo burial practices may have
constrained local morphological requirements, highlighting differential
design intent.

A Procrustes ANOVA was used to test whether a significant difference
exists in Gahagan biface shape by context (RRPP = 10,000; Rsq = 0.21033;
Pr(\textgreater F) = 1e-04) (Figure 6), followed by a second to assess
differences in Gahagan biface (centroid) size by context (RRPP = 10,000;
Rsq = 0.01138; Pr(\textgreater F) = 0.3933).

\begin{figure}

{\centering \includegraphics[width=1\linewidth]{img/fig05} 

}

\caption{The difference between Gahagan bifaces from cache and individual contexts is characterized by a narrower base, and a reduction in the flexuous blade shape (recurve) for bifaces found atop or alongside Caddo individuals. In the comparison of contexts, cache is presented in gray, and individual in black.}\label{fig:mshape.bpractice}
\end{figure}

\hypertarget{discussion}{%
\section{Discussion}\label{discussion}}

Temporal differences in Gahagan bifaces articulate with size rather than
shape at the Mounds Plantation site, no differences in shape or size
were found to occur in Gahagan bifaces recovered from F119 and F134 at
the George C. Davis site, and bifaces recovered from \emph{cache} and
\emph{individual} contexts exhibit a significant difference in shape
(see Figure 6).

The hypothesis that \emph{Gahagan bifaces selected for inclusion in
Caddo burials differ in morphology through time at the Mounds Plantation
site} was tested using two samples of Gahagan bifaces from Caddo burial
contexts at the Mounds Plantation and George C. Davis sites where
stratigraphy dictates differing (earlier/later) temporal positions.
There is an important cultural distinction between archaeological
contexts at Mounds Plantation and George C. Davis, in that all of the
Gahagan bifaces from Mounds Plantation were recovered \emph{in
association with an individual} (Burials 1, 2, 5, and 8) \citep{RN8174},
and all but four of the Gahagan bifaces from George C. Davis articulate
with \emph{caches included along the northern periphery of group
burials} (Features 119 and 134) \citep{RN5746, RN8186}.

Previous studies have demonstrated a significant difference in shape
between Gahagan bifaces recovered from the Mounds Plantation and George
C. Davis sites \citep{RN8154}. In burials at Mounds Plantation, the
Caddo were \texttt{selecting} for Gahagan bifaces that were
significantly smaller in later contexts; a pattern that is not present
at George C. Davis.

Chronometric dates associated with contexts yielding Gahagan bifaces
leave much to be desired, and it is not possible at this time to order
the burial contexts sequentially using chronometric or relative dating
methods. Results raise substantive questions regarding the biotic,
abiotic, and/or cultural variables that were driving the temporal shift
in size at Mounds Plantation. Might this size difference articulate with
a shift in trading or exchange-based relationships with central Texas
groups, and/or might the shift be related to a shift in functional use?

At both sites, shape remains consistent and does not differ between
contexts, indicating an established \texttt{shape\ preference} that may
have shifted at Mounds Plantation due to cyclical differences in the
variable social mechanisms associated with raw material procurement. It
may also be the case that the size difference was related to signaling,
as well as the intended audience. For instance, larger bifaces selected
for inclusion with caches may suggest that the intended audience was the
group; representative of a potentially ostentatious display. The smaller
Gahagan bifaces from individual burials may have been personal items
belonging to the deceased, intended to signal status among personal
contacts.

\hypertarget{gahagan-biface-morphology-differs-by-burial-context}{%
\subsection{Gahagan biface morphology differs by burial
context}\label{gahagan-biface-morphology-differs-by-burial-context}}

The hypothesis that \emph{Gahagan bifaces included in Caddo burials as a
cache differ in morphology from those placed atop or alongside an
individual} was tested using Gahagan bifaces that articulate with two
distinct Caddo burial practices; one interred as part of a \emph{cache},
and the other placed atop or alongside \emph{individuals}. Burial
practices may have been constrained by local morphological requirements,
highlighting aspects of differential design intent.

Results highlight a significant difference in shape--but not size--for
Gahagan bifaces included with Caddo burials. Contextual differences
suggest two Caddo burial traditions associated with Gahagan bifaces; one
more prevalent in the \texttt{northern\ behavioral\ region} where
Gahagan bifaces were placed \emph{atop or alongside an individual}, and
one more prevalent in the \texttt{southern\ behavioral\ region} where
Gahagan bifaces were \emph{included as a cache offering along the
northern periphery of the burial}. Each burial tradition appears to have
been bounded by its' own \texttt{community\ of\ practice} relating to
both the placement of the bifaces, and the design (shape) of the Gahagan
bifaces used in each context.

While the northern \texttt{community\ of\ practice} appears to have been
in operation at the same time as that of the southern
\texttt{community\ of\ practice}, evidenced by individuals buried at
Gahagan Mound and George C. Davis with Gahagan bifaces placed atop or
alongside them, the inverse is not currently supported by the
archaeological record. This raises questions regarding whether the
northern burial tradition predates that of the south, and/or whether the
\texttt{spatial\ boundary} may have been permeable, but in only one
direction.

\hypertarget{conclusion}{%
\section{Conclusion}\label{conclusion}}

\hypertarget{acknowledgments}{%
\section*{Acknowledgments}\label{acknowledgments}}
\addcontentsline{toc}{section}{Acknowledgments}

We extend our gratitude to the Caddo Nation of Oklahoma, the Williamson
Museum at Northwestern State University, the Louisiana State Exhibit
Museum, the Texas Archeological Research Laboratory at The University of
Texas at Austin, the Brazos Valley Museum of Natural History, the Texas
Parks and Wildlife Department, and the Sam Noble Oklahoma Museum of
Natural Science for the requisite permissions and access needed to
generate 3D scans of the Gahagan bifaces. Thanks to Harry J. Shafer,
Hiram F. (Pete) Gregory, Christian S. Hoggard, and David K. Thulman for
their comments on the analyses of Gahagan biface shape.

RZS extends his gratitude to Christian S. Hoggard and David K. Thulman
for their thoughtful comments and constructive criticisms of the
landmarking protocol used in this study
(\href{https://github.com/aksel-blaise/gahaganmorph2/blob/master/analysis/landmarking-protocol.md}{\texttt{LM3d1}}),
as well as the landmarking protocol for Gahagan bifaces that will be
used in the next iteration of these analytical efforts
(\href{https://seldenlab.github.io/gahaganmorph.3/landmarking-protocol-3d2.html}{\texttt{LM3d2}});
to Martin Hinz for fielding questions related to the \texttt{oxcAAR}
package and Derek Hamilton for his guidance with the chronological
models; and to Dean C. Adams, Michael L. Collyer, Emma Sherratt, Lauren
Butaric, and Kersten Bergstrom for their constructive criticisms,
general comments, and suggestions throughout the development of this
research program.

\hypertarget{funding}{%
\section*{Funding}\label{funding}}
\addcontentsline{toc}{section}{Funding}

Components of this analytical work flow were developed and funded by a
Preservation Technology and Training grant (P14AP00138) to RZS from the
National Center for Preservation Technology and Training (NCPTT), and
additional grants to RZS from the Caddo Nation of Oklahoma, National
Forests and Grasslands in Texas (15-PA-11081300-033) and the United
States Forest Service (20-PA-11081300-074). Funding to scan the Gahagan
bifaces at the Williamson Museum at Northwestern State University,
Louisiana State Exhibit Museum, Texas Archeological Research Laboratory
at The University of Texas at Austin, and Sam Noble Oklahoma Museum of
Natural Science was provided to the RZS by the Heritage Research Center
at Stephen F. Austin State University.

\hypertarget{data-management}{%
\section*{Data management}\label{data-management}}
\addcontentsline{toc}{section}{Data management}

The analysis code associated with this project can be accessed through
the supplementary materials
(\url{https://seldenlab.github.io/gahaganmorph.3/}) or the GitHub
repository (\url{https://github.com/seldenlab/gahaganmorph.3}), both of
which are digitally curated on the Open Science Framework
\href{https://osf.io/y7b39/}{DOI: 10.17605/OSF.IO/Y7B39}. The
reproducible nature of this undertaking provides a means for others to
critically assess and evaluate the various analytical components
\citep{RN8312,RN8313,RN8299}, which is a necessary requirement for the
production of reliable knowledge.

Reproducibility projects in \href{https://osf.io/ezcuj/}{psychology} and
\href{https://www.cos.io/rpcb}{cancer biology} are impacting current
research practices across all domains. Examples of reproducible research
are becoming more abundant in archaeology
\citep{RN8207,RN8965,RN8154,RN8318,RN9364,RN11064}, and the next
generation of archaeologists are learning those tools and methods needed
to reproduce and/or replicate research results \citep{RN10760}.
Reproducible and replicable research work flows are often employed at
the highest levels of humanities-based inquiries to mitigate concern or
doubt regarding proper execution, and is of particular import should the
results have---explicitly or implicitly---a major impact on scientific
progress \citep{RN10761}.

\bibliographystyle{tfcad}
\bibliography{interactcadsample.bib}


\input{"appendix.tex"}


\end{document}
